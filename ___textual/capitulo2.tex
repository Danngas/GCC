\chapter{REVISÃO BIBLIOGRÁFICA}{}
\label{cap:02}
O Quadro \ref{tab:q1} apresenta uma descrição com os principais estudos bibliográficos utilizados como base para o presente trabalho.

\begin{quadro}
\caption{Principais referências bibliográficas utilizadas} \label{tab:q1}\\
\end{quadro}
\begin{center}
    \begin{longtable}{|p{2.5cm}|p{1.5cm}|p{4cm}|p{6cm}|}
        %\caption{Exemplo de longtable}\\
        \hline
        \centering \textbf{Autor} & \centering  \textbf{Tipo} & \centering  \textbf{Objetivo} & \centering \textbf{ Resultados} \\
        \hline
        \endfirsthead
       % \multicolumn{4}{c}%
      % {\tablename\ \thetable\ -- \textit{Continuação da tabela}} \\
       % \hline
       % \centering \textbf{Autor} & \centering  \textbf{Tipo} & \centering  \textbf{Objetivo} & \centering \textbf{ Resultados} \\
       % \hline
      %  \endhead
        \hline \multicolumn{4}{r}{\textit{Continua na próxima página}} \\
        \endfoot
        \hline \multicolumn{4}{r}{\textit{Fim do Quadro}} \\
        \endlastfoot
       \centering \citeonline{correia2016automaccao} & \centering Artigo & 
       Este trabalho apresenta uma proposta de desenvolvimento de um protótipo de baixo custo de aquisição (plataforma \textit{Arduino}\footnote{Arduino - Placa de prototipagem eletrônica de código aberto.}) para monitoramento e controle automático da irrigação, com acionamento remoto via aplicativo.

                & 
       Após testes de usabilidade com seis voluntários, 
        evidenciou-se uma boa receptividade do sistema, 
        sendo que estes não encontraram dificuldades em 
        operar o sistema.
        Houve uma redução do consumo de água 
       durante as irrigações em torno de 26,80\% no modo 
        automático do sistema.

               \\ \hline 
               \centering \citeonline{lima2019desenvolvimento} & \centering Artigo & Este artigo propõe 
            um projeto de controle automático para sistema de irrigação 
            utilizando tecnologias \ac{IoT}.
 & Com base nos testes realizados o sistema de controle é seguro. Trocando a antena do HC12 por uma antena dipolo com maior ganho foi possível alcançar a distância desejada. Dessa forma, o projeto de controle automático para sistema de irrigação desenvolvido é eficaz, eficiente e economicamente viável. \\ \hline 
               \centering \citeonline{velasco2019internet} & \centering Artigo & O principal objetivo do desenvolvimento é fornecer um protótipo de modelo com tecnologia combinada no monitoramento de uma horta para sistemas de irrigação. & Após vários testes sobre a capacidade do sistema, o aplicativo \ac{IoT} para sistemas de irrigação inteligente atendeu às necessidades dos agricultores que utilizam dispositivos \ac{IoT}. \\ \hline 
               \centering \citeonline{pisanu2020prototype} & \centering Artigo & Projetar, desenvolver e construir um protótipo de uma
plataforma eletrônica de baixo custo para monitoramento de ambiente de estufa em tempo real.
 & O protótipo da plataforma eletrônica para monitoramento do ambiente de estufa foi projetado para ser modular, tendo obtido êxito em várias combinações com \textit{firmwares}\footnote{firmware - Programa embarcado em \textit{hardware}} otimizados. \\ \hline
        \centering \citeonline{widyawati2020design} & \centering Artigo & Este artigo apresenta um sistema de monitoramento das condições ambientais em uma estufa usando a tecnologia da  \ac{IoT}. & Os resultados dos testes durante sete dias mostraram que o protótipo \ac{IoT} para sistemas de monitoramento de estufas foi capaz de funcionar bem. Isso é indicado pela porcentagem média de dados armazenados com sucesso em 99,76\% e a porcentagem média de perda ou duplicação de dados de 0,24\%.  \\ \hline 
        \centering \citeonline{friha2021internet} & \centering Artigo & Este artigo apresenta uma revisão abrangente das tecnologias emergentes para a agricultura inteligente baseada na Internet das Coisas. & Por meio de extensa pesquisa e análise conduzidas, foi possível classificar os aplicativos \ac{IoT} para agricultura inteligente em sete categorias, incluindo monitoramento inteligente, gerenciamento de água, aplicativos agroquímicos, gerenciamento de doenças, colheita inteligente, gerenciamento da cadeia de suprimentos e gerenciamento inteligente. \\ \hline 
        \centering \citeonline{marques2021plataforma} & \centering Artigo & Este artigo tem como objetivo desenvolver um sistema de coleta de 
dados, de baixo custo, para a obtenção de parâmetros relacionados à luminosidade, umidade do solo, 
umidade do ar e temperatura em ambiente agrícola.
 & A plataforma \textit{Arduino} e os sensores acessórios, mostraram-se perfeitamente aplicáveis para a aquisição e
armazenamento de dados em casas de vegetação. O protótipo de registrador de dados desenvolvido apresentou
redução de custo de 600 até 3000\% em relação aos componentes disponíveis no mercado com
funcionalidades semelhantes.
 \\ 


    \end{longtable}
\end{center}

O presente trabalho	desenvolveu um controle de irrigação no ambiente de uma estufa, semelhante ao apresentado em \citeonline{widyawati2020design}. Esta abordagem permite um maior controle das variáveis do ambiente. Em contrapartida, em \citeonline{lima2019desenvolvimento}, só é possível controlar o nível de umidade do solo, limitando as possibilidades de manipulação das condições microclimáticas, entretanto, simplificando a aplicação em campo. 

Dessa forma, o monitoramento e controle da planta foi realizado de maneira remota, por meio de \textit{software} computacional, semelhante ao apresentado em \citeonline{correia2016automaccao}, mas, o microcontrolador selecionado foi o PIC 18F4550, já que o mesmo possui capacidade de processamento adequada à aplicação.

A interface humano-máquina pode ser acessada por plataforma \textit{desktop} e por sistemas operacionais \textit{Android}\footnote{Android - Sistema operacional desenvolvido para dispositivos móveis.} ou IOS\footnote{IOS - Sistema operacional desenvolvido exclusivamente para dispositivos da \textit{Apple}}, possibilitando o monitoramento e controle do sistema de qualquer local com acesso à internet, facilitando o acompanhamento e acesso aos dados em tempo real.
