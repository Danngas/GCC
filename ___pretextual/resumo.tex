% o texto deve começar imediatamente na próxima linha a fim de evitar identação
O trabalho apresentado expõe um sistema para a classificação de doenças em plantações de bananeiras utilizando redes \ac{RNC}. O objetivo principal é auxiliar produtores de banana na identificação precoce de doenças como Sigatoka, que prejudicam a produção. A metodologia envolve a coleta e processamento de imagens de folhas de bananeiras, essas são classificadas em saudáveis ou infectadas, usando a base de dados \ac{BananaLSD}. Técnicas de processamento de imagem e aprendizado profundo foram aplicadas para aumentar a precisão do modelo. O treinamento e a avaliação foram conduzidos com diferentes parâmetros, resultando em um modelo melhor de classificar as doenças. O trabalho também explora os desafios relacionados ao incremento de novas doenças e ao ajuste de parâmetros e otimização de desempenho, oferecendo um cronograma de atividades futuras para aprimoramento do sistema. 

\palavraschave{Redes neurais convolucionais, classificação de doenças, Sigatoka, aprendizado profundo, processamento de imagens.}
% ATENÇÃO! Obrigatório um espaço entre a última linha do resumo e o comando \palavraschave{...}
